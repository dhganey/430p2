\documentclass[]{article}

%packages
\usepackage[top=1in, bottom=1in, left=1in, right=1in]{geometry} %adjust margins. TODO: hack to fix large top margin
\usepackage{setspace} %allows doublespacing, onehalfspacing, singlespacing
\usepackage{enumitem} %for continuing lists
\usepackage{titling} %for moving the title
\usepackage[normalem]{ulem} %for underlining
\usepackage{graphicx} %for inserting images
\usepackage{listings} %for source code

\renewcommand{\thesubsection}{\thesection.\alph{subsection}}

\begin{document}
\lstset{language=C++}

\begin{spacing}{.4}
\setlength{\droptitle}{-7em}
\title{CSE 430 Project 2}
\author{David Ganey}
\date{October 31st, 2014}
\maketitle
\end{spacing}

%Content
\section{Design}
For this project, I decided to store the entire file as a vector of strings. To simplify the program and to avoid typing \texttt{std::vector<std::string>} repeatedly, I created a typedef for this structure called \texttt{strvec}. Reading the file is trivial--a loop using a \texttt{stringstream} and the \texttt{getline} function allows each line to come in as a separate string, which is then pushed into the vector. Trim functions are used to remove all whitespace. This decreases readability of the output file, but allows for the assumption that the "meat" of each line begins at character 0.
\newline \newline
After generating this vector, the the program simply iterates through it looking for \texttt{\#pragma} directives. TODO THIS MIGHT CHANGE OR SOMETHING. EXPAND THIS
TODO HOW did you handle each clause? did you even do that?
\newline \newline
The program reads the input file only once but processes the \texttt{strvec} containing the input many times.
\newline \newline
any other decisions or difficulties

\end{document}